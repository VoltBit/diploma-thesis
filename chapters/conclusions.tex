\chapter{Conclusions}

To sum up all the aspects presented in the previous chapters, the framework proposal is more of an architectural proposal accompanied by a functional but limited software representation. The solution focuses on the fundamental requirement - being able to validate a Linux distribution based on a specification - and offers the functionality through a flexible model and simple design. The abstraction of specifications, quantification of compliance and the extensive use of open-source tools to create a convenient and useful framework is the main achievement of the project.

The architecture is simple, based on modules defined by a specific role. It is flexible enough to accommodate various testing setups and, as discussed in the previous chapter, has the potential to offer a scalable solution with limited setup and management effort. The technical implementation uses lightweight, efficient and extensible open source projects that are already used in production systems.

In conclusion, Ellida is an incomplete project, but offers the foundation for a product that, when mature enough, can be used in the open-source community and beyond. It has yet to be proven in a real world environment, but has the potential to fulfil its purpose - alleviate the development efforts through automation. 
